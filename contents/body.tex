%% body.tex
%% Copyright 2022 skyleaworlder
%
% This work may be distributed and/or modified under the
% conditions of the LaTeX Project Public License, either version 1.3
% of this license or (at your option) any later version.
% The latest version of this license is in
%   http://www.latex-project.org/lppl.txt
% and version 1.3 or later is part of all distributions of LaTeX
% version 2003/12/01 or later.
%
% This work has the LPPL maintenance status "maintained".
%
% This Current Maintainer of this work is skyleaworlder.
%
% This work consists of all the *.tex and *.sty files in
%   https://github.com/TJ-CSCCG/Tongji-Beamer
\section{系统分析}
    % “图片文字并排” 示例
    \begin{frame}{内卷行为归因系统}
        \begin{columns}
            \column{.35\textwidth}
            \begin{figure}
                \centering \includegraphics[width=2in]{contents/figure/factor-analyzer.png}
                \caption{归因系统架构}
                \label{fig:factor-analyzer}
            \end{figure}

            \column{.6\textwidth}
            \begin{itemize}
                \item 校内监控系统:接入校内系统,使用过往录像作为归因系统模型训练数据集。
                \item 特征提取器:项目提出了数十种内卷特征类型。提取器要求人工为部分数据集标记标签。
                \item 内卷归因器:阅读历史文献,得出数十种内卷成因,构建内卷特征到成因的映射函数。
                \item 报告持续生成器:开发该系统的报告生成器,提升用户体验。
            \end{itemize}
        \end{columns}
    \end{frame}

    % “图片文字垂直” 示例
    \begin{frame}{内卷识别器}
        \begin{figure}
            \centering
            \includegraphics[width=.4\textwidth]{contents/figure/data-detector.png}
            \caption{识别器架构}
            \label{fig:data-detector}
        \end{figure}

        \begin{itemize}
            \item \small 复用归因系统中有关图像处理的 Phase,编写软硬件接口,对接校内监控系统终端设备。
            \item \small 识别器特征提取结果将上传至我校私有云上,方便监管与处理。
        \end{itemize}
    \end{frame}

    % “多图片文字混合” 示例
    \begin{frame}{内卷信息收集与行为防范系统}
        \begin{columns}
            \column{.4\textwidth}
            \begin{figure}
                \centering
                \includegraphics[width=\textwidth]{contents/figure/data-detector.png}
                \caption{识别器架构}
                \label{fig:data-detector-2}
            \end{figure}

            \column{.3\textwidth}
            \begin{figure}
                \centering
                \includegraphics[width=\textwidth]{contents/figure/collector-part-1.png}
                \caption{信息收集系统}
                \label{fig:collector-part-1}
            \end{figure}

            \column{.3\textwidth}
            \begin{figure}
                \centering
                 \includegraphics[width=\textwidth]{contents/figure/collector-part-2.png}
                \caption{行为防范系统}
                \label{fig:collector-part-2}
            \end{figure}
        \end{columns}

        \small 信息收集系统使用一个服务作为所有监控终端设备的 “注册中心”,提供鉴权、心跳检查等能力。行为规范系统复用了归因系统中有关因素分析的 Phase,以此作为决策树的判断基础。
    \end{frame}


\section{合理性分析}
    % “数学” 与 “公式” 示例
    \begin{frame}{专有名词释义}
        \begin{definition}[永夜势力]
            考虑一所高校,其中永夜村民的数量除以学生总数的值。
        \end{definition}
        \begin{theorem}[永夜定理]
            考虑一所高校,在不存在干涉的情况下,永夜势力在小于 1 前,恒呈指数型增长。
        \end{theorem}
        \begin{theorem}[永夜方程]
            考虑一所高校,人为干涉存在 x 的概率无效。x 是下列方程的解,其中 n 为全校学生总数。
        \end{theorem}

        \begin{equation}
            \begin{vmatrix}
            \begin{bmatrix}
                    x & \Phi(\pi) \\
                    \sum_{i = 1}^n xi & \exp(xi)
            \end{bmatrix}
            \begin{bmatrix}
                    1 & \cdots & n \\
                    x\Phi(e) & \cdots & x\Phi^n(e^n)
            \end{bmatrix}
            \end{vmatrix} = \dfrac{\pi}{x}
        \end{equation}
    \end{frame}

    % “简单公式” 示例
    \begin{frame}{负载分析}
        系统在设计时进行了计算下沉,分离图像处理的多个步骤,实现计算的高效性。

        假设 $id \in [1850001, 1859999]$,识别器与分析器共有 $m$ 个机器,第 $i$ 个机器负责 $[1850001 + i \times \frac{9999}{m}, 1859999 + (i + 1) \times \frac{9999}{100}]$ 个单位。

        识别层机器会发送分析层机器需要的特定记录到存储中。存储中每个节点大小为 16 KB,由于一条记录占据 4 00B 的空间,有:

        $$
        Node = \dfrac{16 \times 1024 B}{400 B} \approx 16
        $$

        假设 $t$ 为一次操作所花费的时间,设单表大小为 $M$。在不进行计算分离时,消耗的时间 $T = Mt$。但在分离过后,全过程时间为:

        $$
        T' = \dfrac{M}{m} + \dfrac{9999}{m} \times t < Mt, \quad if\ m > \dfrac{1}{t} + \dfrac{9999}{M}
        $$
    \end{frame}


\section{组件设计}
    % “表格” 示例
    \begin{frame}{决策系统设计}
        \begin{table}[]
            \centering
            \begin{tabular}{c|c|c}
            \textbf{运行时间} & \textbf{实验组} & \textbf{对照组(单位 1)} \\
            \hline
                设计模式使用 vs. 不使用 & 1.1 & 1 \\
                Julia vs. Python & 0.7 & 1 \\
                负载均衡集群 vs. 单机 & 0.5 & 1 \\
                集群分布式缓存 vs. 无缓存 & 0.8 & 1 \\
            \end{tabular}
            \caption{各技术对决策系统的影响}
            \label{tab:tech-strategy}
        \end{table}

        \begin{columns}
            \small
            \column{.5\textwidth}
            \begin{itemize}
                \item 使用 Strategy 设计模式。
                \item 使用 Julia 高效计算矩阵乘法。
                \item 对归因系统暴露幂等性 API,使用计算集群增强计算能力。
            \end{itemize}

            \column{.5\textwidth}
            \begin{itemize}
                \item 采用负载均衡,平衡集群中各处理机的负载。
                \item 采用分布式缓存技术,加快集群整体对高频特殊输入的反应。
            \end{itemize}
        \end{columns}
    \end{frame}

    % “引用” 示例
    \begin{frame}{通知系统设计}
        著名哲人 skyleaworlder 有言:“内卷是可以被扼杀在摇篮中的。"\cite{involution} 只要在检测到疑似内卷行为后做到及时响应,由专业技术人员发起干涉,就有很相当概率阻止永夜势力的扩张。

        通知系统需要具备以下几个方面特性:
        \begin{enumerate}
            \item 即时性。内卷极易滋生壮大,通知专业技术人员的时间应该尽可能短。
            \item 高可用性。内卷行为来势汹汹,需要保证通知系统推送功能在高并发环境下的高可用性。
        \end{enumerate}

        自然卷说过:“内卷是凡人在挣扎中写下的血与泪的史诗。"\cite{undergrads_inv} 这份 “扭曲的美” 不应存于世间。
    \end{frame}
