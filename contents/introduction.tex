%% introductioin.tex
%% Copyright 2022 skyleaworlder
%
% This work may be distributed and/or modified under the
% conditions of the LaTeX Project Public License, either version 1.3
% of this license or (at your option) any later version.
% The latest version of this license is in
%   http://www.latex-project.org/lppl.txt
% and version 1.3 or later is part of all distributions of LaTeX
% version 2003/12/01 or later.
%
% This work has the LPPL maintenance status "maintained".
%
% This Current Maintainer of this work is skyleaworlder.
%
% This work consists of all the *.tex and *.sty files in
%   https://github.com/TJ-CSCCG/Tongji-Beamer
\section{引言}
    \begin{frame}
    % “无序列表” 与 “有序列表” 使用
    \frametitle{引言}
        \footnotesize
        \begin{block}{项目背景}
            \begin{itemize}
                \item 二十一世纪二十年代初,各大高校学生被卷入到一场声势浩大、旷日持久的内卷运动中。后世称之为 “永夜行动”,参加该类活动的学生被称为 “永夜村民”。
                \item 济勤学堂作为我校八大学堂之一,聚集了人工智能、大数据、计科、软工、信安、自动化等新工科专业,吸引了大批永夜村民。
                \item 当今时代不再内卷。但内卷仍是人们心中永远的痛。防范 “永夜村势力” 抬头,是当代每个优秀新青年的重要责任。
            \end{itemize}
        \end{block}

        \begin{block}{项目要素}
            \begin{enumerate}
                \item 开发内卷行为归因系统,通过历史数据分析得出二十年代初永夜行动的直接与间接原因。
                \item 开发内卷识别器,并在全校范围内构建内卷识别天网。
                \item 开发内卷信息收集与行为防范系统,避免永夜行动的重演。
            \end{enumerate}
        \end{block}
    \end{frame}

